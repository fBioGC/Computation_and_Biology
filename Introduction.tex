\section{Introducción}
 
Al ser esta la primera asignatura en la carrera con carácter explícitamente computacional, hemos querido comenzar agregando un poco de contexto sobre la relación existente entre la Biología y la Computación.
Para ello, haremos una revisión muy breve sobre la historia de cómo la primera ha ido encontrando en esta última una herramienta cada vez más útil, incluso imprescindible, a la hora de resolver las cuestiones fundamentales en el estudio de la vida.
En su elaboración, el mayor reto al que nos enfrentamos es la diversidad de temas de interés.
La Biología es una ciencia muy amplia, que va desde los estudios más holísticos en la ecología y la conservación, hasta los más reduccionistas como la caracterización de una proteína.
Para cada una de estas áreas se pudiera hacer un relato independiente pero en casi todos encontraríamos un patrón similar.
La fracción de cualquier estudio biológico dependiente de las habilidades computacionales del investigador ha ido creciendo sostenidamente; ya sea durante la recolección de los datos, su análisis, su almacenamiento o su publicación \cite{gauthierBriefHistoryBioinformatics2019, cameronBriefHistorySynthetic2014}.
 
 
Por problemas de tiempo, y aceptando que dejaremos fuera muchos casos interesantes, en este documento nos centraremos en una historia en particular: el desarrollo y consolidación del Dogma Central de la Biología Molecular.
La justificación de nuestra selección es que dicho principio es uno de los conceptos fundamentales de la Biología moderna y encuentra aplicación en todas las áreas de la misma.
A grandes rasgos, este establece la popular noción de que en los sistemas vivos el flujo de información comienza típicamente por la secuencia de ADN (ácido desoxirribonucleico) y luego se ramifica escalonadamente por varios niveles.
Otro rasgo importante es que este proceso es unidireccional.
O sea, los cambios fenotípicos no se propagan hacia la cumbre de la jerarquía afectando a la secuencia de ADN \cite{crickCentralDogmaMolecular1970}.
Posteriormente profundizaremos más sobre el tema pero por ahora, comenzaremos un poco antes de que todo ese conocimiento se estableciera.
